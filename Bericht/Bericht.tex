\documentclass[12pt,a4paper]{article}
\usepackage[utf8]{inputenc}
\usepackage[german]{babel}
\usepackage{amsmath}
\usepackage{amsfonts}
\usepackage{amssymb}
\usepackage{graphicx}
\usepackage{enumitem}
\usepackage{gensymb}
\usepackage{wrapfig}
\usepackage{tikz}
\newcommand*\circled[1]{\tikz[baseline=(char.base)]{
            \node[color=blue,shape=circle,draw,inner sep=2pt] (char) {#1};}}
\author{Philipp Oldenburg, Patrick Zumsteg, Simon Wallny}
\title{Gruppe „Denial of Service“}
\date{Frühjahrssemester 2015}
\begin{document}
\maketitle
\tableofcontents
\section{Vorwort}
Das Projekt „Denial of Service“, entstand im Rahmen der Vorlesung „Internet-Technologien“ an der Universität Basel im Frühjahrssemester 2015.\hfill\\
Die bearbeitende Gruppe besteht aus Philipp Oldenburg, Patrick Zumsteg und Simon Wallny.
\newpage
\section{Angriffs-Strategien}
\subsection{Applikationsebene}
SlowLoris
\subsection{Transportebene}
Syn Flood
\subsection*{Netzwerkebene}
ICMP Flood
\section{Verteidigungs-Strategien}
Stecker ziehen.
\section{Bekannte Anwendungen}
LOIC\\
OWASP

\end{document}